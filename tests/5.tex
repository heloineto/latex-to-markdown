% After typing in the commands to LaTeX (which are the instructions preceded by
% the backslash character) and the text of a sample paper, save them in a file
% with a name ending in .tex, like paper.tex. You can then type latex paper.tex
% and the typesetting program will run on your file of commands, producing a file
% ending in .dvi, which is the file that can be sent to a laserprinter (like valkyr,
% in Margaret Jacks Hall). (If there are any errors in your file of commands,
% you will be given a message which is usually impossible to interpret. Typical
% errors involve forgetting the right number of closing brackets or delimiters
% like & in example sentences. When LaTeX gives an error message and then asks
% what to do, possible options are to type x to quit and try to find the error
% in the emacs file, or press <RETURN> to try to continue and find the error by
% looking at the output document.) To print the file, you type lpr -Pvalkyr

\documentclass[12pt]{article}
\usepackage{lingmacros}
\usepackage{tree-dvips}

\title{Esse eh o ultimo}
\author{bem interessante}

\begin{document}
\maketitle
\chapter{Notes for My Paper}

\section{Notes for My Paper}

Don't forget to include examples of topicalization.
They look like this:

\subsection{How to handle topicalization}

I'll just assume a tree structure like.

\subsection{Mood}

Mood changes when there is a topic, as well as when
there is WH-movement.  is the mood when
there is a non-subject topic or WH-phrase in Comp.
is the mood when there is a subject topic
or WH-phrase.

\end{document}