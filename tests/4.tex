\documentclass[12pt]{article} 
\usepackage[brazil]{babel} 


\title{Continue a testar \LaTeX} 
\author{Douglas Adams}


\begin{document} 
\maketitle
As frases a seguir veem de muita reflexão!

\chapter{\bf{O guia}}

\it{“Existe uma teoria que diz que, se um dia alguem descobrir exatamente para que serve o Universo e por que ele esta aqui, ele desaparecera instantaneamente e sera substituído por algo ainda mais estranho e inexplicavel. \bf{Existe uma segunda teoria que diz que isso ja aconteceu.}”}

\section{introduction}
 “Um dos maiores problemas encontrados em viajar no tempo nao eh vir a se tornar acidentalmente seu proprio pai ou mae. Nao ha nenhum problema em tornar-se seu proprio pai ou mae com que uma familia de mente aberta e bem ajustada nao possa lidar. Tambem nao ha nenhum problema em relacoo a mudar o curso da historia – o curso da historia nao muda porque todas as peças se juntam como num quebra-cabeca. Todas as mudancas importantes ja ocorreram antes das coisas que deveriam mudar e tudo se resolve no final. \underline{O problema maior eh simplesmente gramatical}.”

\subsection{methodology}
“O Guia do Mochileiro das Galaxias ja substituiu a grande Enciclopedia Galactica. Em primeiro lugar, eh ligeiramente mais barato; em segundo lugar, traz na capa, em letras garrafais e amigaveis, a frase \bf{NÃO ENTRE EM PANICO}.”

\subsection{conclusion}
“Realidade eh frequentemente imprecisa.”

\end{document}